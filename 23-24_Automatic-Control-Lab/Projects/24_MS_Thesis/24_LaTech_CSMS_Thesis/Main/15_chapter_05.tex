\chapter{CONCLUSION}\label{chap:conclusion}

%Discussion
The results of our experiment in chapter~\ref{chap:fusion} indicate that vision sensing is a promising addition to olfaction sensing in ROSL research. The resutls of chapter~\ref{chap:LLM} indicate that multi-modal reasoning is also a promising approach for zero-shot ROSL navigation. The experimental setup presented mimics indoor environments with obstacles and odor sources. Therefore, the results can be generalized to other real-world indoor odor source localization scenarios, such as detecting indoor gas leaks in office or household settings with obstacles and potential gas sources. It is also feasible to extend the proposed method to outdoor applications, such as detecting wildfire locations using both vision (flame detection) and olfaction (smoke or other fire-related gases).

The significance of the proposed work can be summarized as follows:
\begin{itemize}
  \item {Integration of vision and olfaction in odor source localization tasks:} Our proposed navigation algorithm integrates both vision and olfaction in odor source localization tasks. Compared to traditional Olfaction-Only Navigation algorithms, including bio-inspired methods \cite{lopez2011moth}, engineering-based methods \cite{luong2023odor, zhu2020novel}, and machine-learning-based methods \cite{kim2019source, hu2019plume}, the addition of vision pushes the boundaries of current ROSL navigation algorithms;
  \item {Zero-shot vision processing for odor source localization:} The proposed multi-modal reasoning-based can process vision information without the need for prior training. Furthermore, the proposed navigation algorithm is general in nature. By changing the navigation objective only, the same navigation algorithm can be used for different robot navigation tasks.
  \item {Odor source localization in complex environments with obstacles:} While most traditional olfactory-based navigation algorithms do not consider obstacles in the search environments (e.g., \cite{lopez2011moth}), our proposed method can guide the robot to find the odor source in complex environments with obstacles. Thanks to the proposed hierarchical control algorithm, the robot can dynamically coordinate among Vision-Based Navigation, Olfaction-Based Navigation, and obstacle avoidance behaviors;
  \item {Real-world experiments and results:} Many prior works (e.g., \cite{hu2019plume}) only validated their algorithms in simulation environments without testing them in real-world settings. However, simulation environments cannot always represent real-world scenarios due to the gap between the simulation and real-world environments. In this work, we implemented the proposed odor source localization algorithm in real-world settings and validated its effectiveness in environments with obstacles and turbulent airflow.
\end{itemize}